\section{Introduction}

Many scientific articles confirm the predictive power of quantitative models of biological systems. Kinetic models are some of the most detailed biological models and contain detailed reaction properties that allow predictions of both reaction fluxes and metabolite levels will be, given a set of model parameters. An early example of a kinetic model is that of the Teusink \emph{et al.} \cite{teusink2000} kinetic model of yeast glycolysis, which with updating of branching reactions was able to predict the \emph{in vivo} flux of half the model reactions to within a factor of 2. In comparison, stoichiometric models of biological systems don't contain enzyme kinetic parameters but instead focus solely on which metabolites participate in which reactions. The iND750 version of the \emph{S.cerevisiae} genome scale model \cite{duarte2004} contains 750 reactions, and the growth rate of the model is simulated by maximising a given objective reaction such as the production of biomass. The growth rate is then defined in the combination of internal reaction fluxes.

Regardless of which type of model used, a common objective is then to analyse, simulate and test the model to observe what predictions the model can make about the corresponding real-life biological system. One avenue of analysis of kinetic biological model is metabolic control analysis, or MCA \cite{fell1997}. One of the key points of MCA is there is not single rate limiting step in a set of reactions, but instead each reaction and metabolite in a system may share a portion of control over the other reaction rates and metabolite levels in the system. As an example the amount of control an enzyme has over a pathway can be estimated by measuring the flux through the system, then reducing the activity of the same enzyme and examining the difference in system flux. This is described in Eq. \ref{fcc}, where $J$ and $Block$ are the fluxes at reaction $m$ and system $i$ respectively. The sign $C_{Block_i}^{J_m}$ is the flux control coefficient, a measure of how sensitive the pathway is to changes in flux through the reaction, $J_{m}$. The flux control coefficients will sum to 1 over a linear pathway as shown in Eq. \ref{summation_theorum}, the summation theorem.

\begin{equation}\label{fcc}
C_{Block_i}^{J_m} = \frac{dJ_m}{J_m}\div\frac{dBlock_i}{Block_i}
\end{equation}

\begin{equation}\label{summation_theorum}
\sum_{i=1}^{n}C_{Block_i}^{J_m} = 1
\end{equation}

An example of how metabolic control analysis can be successfully applied to real biological systems is the examination of tryptophan biosynthesis carried out by Niederberger \emph{et al.} \cite{Niederberger1992}. Using a five-enzyme pathway model involved in the production of tryptophan, it was shown that increasing/decreasing levels of individual enzymes has very little effect on the overall flux through the pathway. This was explained as an effect of the distribution of flux control over the length of the pathway, with no particular enzyme having a large flux control coefficient. However, increasing the concentration of all five enzymes, and therefore the flux through their catalysed reactions, resulted in a synergistic increase in tryptophan flux - much greater than would be expected if the increased enzyme fluxes were only cumulative.

The number of enzyme parameter needed to produce a kinetic model mean that the number of reactions included can be a magnitude smaller than that of genome scale models. The small number of reactions therefore means that the estimation of  control coefficients are limited to the small number of available reactions. For instance, at the time of writing this thesis, there exists no model where it would be possible to derive control coefficients for reactions across different parts of the cell metabolism such as both energy metabolism and amino acid. Existing genome scale models contain reactions contain the majority of the central metabolism for the organism they describe, which solves the problem of limited model size. The restriction of genome scale models is that they do not contain enzyme kinetic data, and so it is not possible to estimate metabolic control coefficients in the same way as kinetic models.

This section highlights the methods and results used in this analysis for trying to estimate the importance and sensitivity of the parts of the genome scale models to simulated perturbations in their activity.

\section{Results}

As discussed, metabolic control analysis can be used to estimate the importance of an object in a kinetic model. In this section I will show similar principles of can be used to estimate the sensitivity of growth rate to the changes in amino acid requirements in biomass. The biomass equation in a flux balance analysis model describes the estimated requirements and stoichiometry of metabolites required for the production of a biomass, which is equivalent to growth. For instance in the biomass reaction in the \emph{S. cerevisiae} model, the requirements include fats such as triglyceride and ergosterol, sugars such as glucose and mannose, as well as smaller metabolites such as sulphate and water. The stoichiometry of the biomass equation is derived from the dry weight biomass of the \emph{S. cerevisiae} cell \cite{duarte2004}. The protein content of a yeast cell is also included in the biomass equation as the twenty individual amino acids that make up protein, with the inclusion of theses amino acids in the biomass equation, the reaction stoichiometries of each can be manipulated to estimate the effect on biomass production.

\begin{figure}
  \subfloat[Supply-demand elasticity for tryptophan]{
    \label{figure:curve_estimation:curve}
  %  \psfrag{change}[B][B]{Relative change in tryptophan requirement}
  %  \psfrag{biomass}[B][B]{Nutrient uptake flux}
    \includegraphics*[width=7cm]{curve_estimation.eps}
  }
  \hfill
  \subfloat[Structure of Tryptophan]{
    \label{figure:curve_estimation:acids}
    \includegraphics*[width=5cm]{tryptophan.eps}
  }
  \hfill
  \subfloat[Comparison of costs for glycine, histidine, and tryptophan]{
    \label{figure:curve_estimation:costs}
    \includegraphics*[width=7cm]{comparison_of_costs.eps}
  }
  \hfill
  \label{figure:curve_estimation}
\end{figure}

Figure~\vref{figure:curve_estimation} illustrates the method use to estimate the cost of an amino acid in the yeast genome scale metabolic model. Figure~\vref{figure:curve_estimation:a} provides and cartoon illustration of the fixation of yeast model parameters. The minimal and maximal boundary on the production of biomass is set to an arbitrary value of 0.05, this has the effect of scaling each estimation of amino acid costs by the same growth rate. Next all nutrient exchange fluxes to have a value of -1000, a much greater excess of the nutrients than the flux the growth rate reaction. The effect of nutrient limitation is then simulated by linear optimisation of the one of the four nutrient uptake fluxes: Glucose, ammonium, sulphate, and phosphate. In plain English this is equivalent to asking "What is the minimal amount of glucose required to produce a growth rate of 0.05, give the structure of yeast metabolism?". Figure\vref{figure:curve_estimation:curve} shows method used to estimate the relationship between amino acid requirement for simulated model growth, and the effect of a given nutrient update, such as glucose and ammonium as illustrated in the figure. In detail the amount of amino acid required for growth, as defined in the model, is changed slightly with either a small increase or small decrease in requirement. The result of this change is then measured in terms of the effect on the uptake flux in question. Figure\vref{figure:curve_estimation:curve} illustrates the effect of an increase or decrease in tryptophan requirement on the uptake fluxes of glucose and ammonium, as this figure shows the relationship of tryptophan with glucose requirement has a steeper slope than that of with nitrogen requirement. The slope of the relationship of tryptophan on the respective uptake flux was interpreted by us as the cost of that amino acid in the given nutrient limitation. This cost is literally the value of $m$ in the linear relationship $y = mx + c$ where $x$ tryptophan uptake requirement, and $y$ is resulting nutrient uptake flux. Figures~\vref{figure:curve_estimation:acids} and \vref{figure:curve_estimation:costs} illustrate the estimated costs for three different amino acids, where Figure~\vref{figure:curve_estimation:acids} shows the chemical structures, and Figure\vref{figure:curve_estimation:costs} illustrates the costs of the same amino acids in two example conditions, glucose (carbon) limitation and ammonium (nitrogen) limitation. Figure~\vref{figure:curve_estimation:costs} typifies how an the estimate cost of an amino acid may vary according to the \emph{in silico} environmental condition in which it is measured. Tryptophan which is an expensive amino acid in glucose limitation, is slightly cheaper in nitrogen limitation. In the reverse of this, histidine which is a relatively cheap amino acid in glucose limitation is correspondingly more expensive, containing three nitrogen atoms, in ammonium limitation. Glycine, a small amino acid with a single hydrogen side chain is relatively cheap in both nutrient limitations.

The cost of producing an amino acid using the above described method was applied to all twenty amino acids in four different nutrient limiting environments: glucose, ammonium, sulphate, and phosphate. The estimated costs for each of these amino acids is illustrated in Figure~\vref{figure:costs_dendrogram_dotplot} along with other costs previously described in the literature. Figure~\ref{figure:costs_dendrogram_dotplot} illustrates that the estimated absolute amino acid costs follow the expected trends, where nitrogen limited cost is proportional to nitrogen content, while only the amino acids that contain sulphur show any cost in the sulphur limited conditions. In terms of the absolute glucose limited costs there, the dendrogram shows that this estimated cost is clustered with the other costs estimated in the literature the closest being the Akashi and Gojobori cost \cite{akashi2002}, also closely related are the Wagner 2005 Respiratory cost \cite{wagner} and the Craig and Weber cost \cite{craig}. A relative measure of cost was also calculated, this being the absolute cost of the amino acid scaled by the amino acid usage in the genome. In terms of the clustering of these costs as shown in Figure~\ref{figure:costs_dendrogram}, the sulphur limited costs are closely correlated with each other, as the carbon and relative and nitrogen relative costs. None of the relative estimated costs show any particularly strong correlation with the manually derived costs previously described in the literature.

\begin{sidewaysfigure}
\centering
\includegraphics*[angle=0,height=13cm]{costs_dendrogram_dotplot.eps}
\caption[Comparison of amino acid cost estimates]{Amino acid cost estimates are shown as bar charts on the left hand side. Each bar chart axis shows the minimum and maximum value of each cost type, rounded to three significant figures. The correlations between costs are compared in a dendrogram on the right hand side computed by complete agglomerative clustering using Spearman's Rank correlation distance between data sets. The illustrated data is shown in Table~\ref{} in Appendix~\vref{appendix:amino_acid_costs}}
\label{figure:costs_dendrogram_dotplot}
\end{sidewaysfigure}

Glucose limited relative and absolute costs were calculated for comparison using the \emph{Escherichia coli iJR904} genome scale model. Figure~\vref{figure:amino_acid_costs_scatterplot} compares both relative and absolute costs for \emph{E. coli} and \emph{S. cerevisiae} models with two costs described in the literature molecular weight \cite{seligmann} and number of ATP and NAPH molecules \cite{akashi2002}.

\begin{figure}
\centering
\includegraphics*[width=13cm]{amino_acid_costs_scatterplot.eps}
\caption[Comparison of the genome scale model derived cost data sets.]{Comparison of estimated amino acid cost with number of ATP and NADPH molecules used in synthesis (left), and molecular weight (right). On the y axis are the amino acid costs estimated using flux balance analysis. Both \emph{S. cerevisiae} and \emph{E. coli} measures are included to illustrate correlation of cost estimates between species. Estimated cost value
es have been been rescaled around their mean value to allow comparisons across species. The trends in each plot are drawn using `loess' smoothing.}
\label{figure:amino_acid_costs_scatterplot}
\end{figure}

Figure~\ref{figure:amino_acid_costs_scatterplot} shows that the both the absolute costs show a small amount of variance in their correlation the example literature described cost measures. Both \emph{E. coli} and \emph{S. cerevisiae} absolute costs measures are particularly well correlated with the Akashi and Gojobori described measure of cost \cite{akashi2002}, however the relationship shows a parabolic relationship. In comparison molecular weight shows a linear relationship with the absolute measures of costs, but shows a greater degree of variation.

The comparison of relative measures of cost illustrates the small degree of correlation with previous estimates of cost. For the both the molecular weight and ATP+NADH cost estimates, neither \emph{E. coli} or \emph{S. cerevisiae} relative cost estimates show any relationship, and furthermore the figure illustrates the disparity in estimated costs between the species models.

\section{Discussion}

This chapter illustrated the methods used to estimate amino acid cost, and further showed the correspondence of these amino acid costs with previously described measures of amino acid cost.

\section{Summary}

Using genome scale models to estimate cost has precedent in the estimates of oxygen related shadow prices in the work of Palsson \emph{et al.} \cite{palsson}. In chapter I have illustrated how similar methods can be used to estimate amino acid costs and how these costs show a high degree of similarity with previous measures of energetic cost, or atomic content. Furthermore new measures of amino acid cost have been estimated taking into consideration the estimated quantity of the amino acid present in biomass.
