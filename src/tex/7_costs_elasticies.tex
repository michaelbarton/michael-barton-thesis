
\begin{table}
\begin{footnotesize}
  \begin{tabular}{ p{1cm} *{6}{p{1.9cm}} }
  \toprule
      & A\&G & C\&W & C\&W & Wager & Wagner & Seligmann \\
      & Energy                 & Energy                & Steps                 & Fermentative        & Respiratory & Weight \\ \midrule
      ala & 11.7 & 12.5 & 1 & 2 & 14.5 & 89.1 \\
      arg & 27.3 & 18.5 & 10 & 13 & 20.5 & 174.2 \\
      asn & 14.7 & 4 & 1 & 6 & 18.5 & 132.1 \\
      asp & 12.7 & 1 & 1 & 3 & 15.5 & 133.1 \\
      cys & 24.7 & 24.5 & 9 & 13 & 26.5 & 121.2 \\
      gln & 16.3 & 9.5 & 2 & 3 & 10.5 & 146.2 \\
      glu & 15.3 & 8.5 & 1 & 2 & 9.5 & 147.1 \\
      gly & 11.7 & 14.5 & 4 & 1 & 14.5 & 75.1 \\
      his & 38.3 & 33 & 1 & 5 & 29 & 155.2 \\
      ile & 32.3 & 20 & 11 & 14 & 38 & 131.2 \\
      leu & 27.3 & 33 & 7 & 4 & 37 & 131.2 \\
      lys & 30.3 & 18.5 & 10 & 12 & 36 & 146.2 \\
      met & 34.3 & 18.5 & 9 & 24 & 36.5 & 149.2 \\
      phe & 52 & 63 & 9 & 10 & 61 & 165.2 \\
      pro & 20.3 & 12.5 & 4 & 7 & 14.5 & 115.1 \\
      ser & 11.7 & 15 & 3 & 1 & 14.5 & 105.1 \\
      thr & 18.7 & 6 & 6 & 9 & 21.5 & 119.1 \\
      trp & 74.3 & 78.5 & 12 & 14 & 75.5 & 204.2 \\
      tyr & 50 & 56.5 & 9 & 8 & 59 & 181.2 \\
      val & 23.3 & 25 & 4 & 4 & 29 & 117.2 \\ \bottomrule
  \end{tabular}
\end{footnotesize}
\label{appendix:table:amino_acid_costs}
\caption[Amino acid costs described in the literature]{Amino acid costs described in the literature. The Akashi \& Gojobori \cite{akashi2002}, Craig \& Weber energy \cite{craig1998}, and the two Wagner \cite{wagner2005} data sets are based on the curation of the number of high-energy molecules used during synthesis, where a defined ratio is used to convert them into a single measures: usually ATP. The Craig \& Weber `steps' measure \cite{craig1998} is based on the number of the number of biosynthetic steps between central metabolism and the produced amino acid. Molecular weight is in Daltons.}
\end{table}
